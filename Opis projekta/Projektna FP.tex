\documentclass[10pt, a4paper]{article}
\usepackage[slovene]{babel}
\usepackage[utf8]{inputenc} %za šumnike
\usepackage{lmodern}
\usepackage[T1]{fontenc}
\usepackage{eurosym}

\begin{document}

\begin{center}
\Huge \textbf{Naslov Skupina 19: Graffiti conjecture 232} \\
\medskip
\Large Urban Merhar, Martin Kokošinek\\
\end{center}

\section{Navodilo}
Računalniško generirana domneva trdi: Če je $G$ enostaven povezan graf, potem
\begin{center}
 $2\gamma_{t}(G) \geq rad(G) + ecc(B)$.
\end{center}
 
Preveri domnevo na različne načine za male in velike grafe. Z uporabo populacijske metahevristike, preveri domnevo v upanju, da jo ovržeš.

\medskip
Nekaj pripomb:

\begin{enumerate}
\item $ecc(v)$ je $ekscentričnost$ od vozlišča $v$. Ekscentričnost od $v$ je razdalja do najbolj oddaljenega vozlišča od vozlišča $v$, i.e., $ max\{d(v,u): u$ je vozlišče na grafu $\}$.
\item $rad(G)$ je $radij$ grafa, t.j., minimum vseh ekscentričnosti vozlišč grafa $G$.
\item $B$ je $obrobje$ grafa $G$, t.j., množica vozlišč z maksimalno ekscentričnostjo.
\item $ecc(S)$ je $ekscentričnost$ množice vozlišč $S$. Definirana je kot: Naj bo $S$ podmnožica množice vozlišč $V$. Razdalja med vozliščem $v$ in množico $S$, definirajmo kot razdaljo od $v$ do najbližjega volišča v $S$. $ecc(S)$ je maksimum razadalj od vozlišča v $V\backslash S$ do množice $S$.
\end{enumerate}


\section{Kratek opis}
$Computer\ generated\ conjectures$ so računalniško ustvarjene domneve. $Graffiti$ je računalniški program, ki generira te matematične domneve oziroma odprte probleme. Računalniši program $Graffiti$ je ustvaril $Siemion\ Fajtlowicz$.\\

V najinem projektu pri predmetu Finančni praktikum si bova ogledala $Graffiti$ $conjecture$ $232$, ki jo bova testirala za majhne in velike grafe v upanju, da najdeva protiprimer. Ideja je, da enačbo zapiševa v programskem jeziku $Sage$ in generirava naključne grafe. Na vsakem od teh grafov pa predpostavko testirava.\\


Že vgrajene funkcije, ki jih bova uporabila v programu:
\begin{enumerate}
	\item $dominating\_ set(total=True, value\_ only=True)$ vrne najmanjšo dominirajočo množico na grafu $G$.
	\item $radius()$ vrne radij grafa $G$.
	\item $eccentricity()$ vrne ekscentričnost vozlišča $v$.
	\item $periphery()$ vrne množico vozlišč iz obrobja grafa $G$.
\end{enumerate}

\subsection{Razlaga pojmov}
\begin{itemize}
\item Dominirajoča Množica $D$: $D$ je množica, kjer je vsako vozlišče iz $G \backslash D$ sosed nekega vozlišča iz $D$.
\item Totalno Dominirajoča množica (TDM): Dominirajoči množici $D$ dodamo pogoj, da so tudi vozlišča dominirajoče množice 
$D$ sosedi vozlišč iz $D$.
\item Totalno Dominirajoče Število (TDŠ): Moč totalno dominirajoče množice grafa $G$.
\item $\gamma_{t}(G)$ je TDŠ grafa $G$.
\end{itemize}

\subsubsection{Populacijska metahevristika}
\textbf{Hevristika} (iz Grščine: 'najdem, odkrijem'): V računalništvu in matematični optimizaciji je visoko-nivojski način reševanja problemov, ko so klasični postopki prepočasni oziroma, ko klasične metode ne vrnejo točnih rezultatov. V zameno za polnost, optimalnost, natančnost, raje pridobimo na časovni zahtevnosti. \\
\textbf{Meta-hevristika} (meta iz Grščine: 'za, onstran') oziroma v prevodu Izčrpna-hevristika: Metahevristika vzame množico rešitev, ki je prevelika za analizo in s pomočjo določenih predpostavk glede optimizacije vrne zadovoljivo rešitev. Ta ni nujno globalno optimalna. \\
\textbf{Populacijska metahevristika}: Ohranjamo večje število kandidatov za rešitev in jih izboljšujemo s pomočjo populacijsih karakteristik. Primer je particle swarm optimization (PSO).

\section{Plan dela}
Zapisati učinkovit algoritem, ki bo za vsak generiran graf preverila lastnosti grafa in posledično domnevo. Za grafe, kjer  domneva ne bi držala pa nam izpiše graf in vrne vrednosti lastnosti, ki so potrebovane v domnevi.

\end{document}